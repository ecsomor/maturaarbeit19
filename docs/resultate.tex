\chapter{Resultate}

\section{Erkenntnisse}
Meine grösste Erkenntnis war, wie schwierig das Abschätzen von Zeiten für Programmierarbeiten ist.

Beim Programmieren ist es anders als beim schreiben eines Buches wo man sich vornehmen kann, an einem Tag eine bestimmte Anzahl an Seiten zu schreiben, so dass man bis zum Ende der Frist genug Seiten hat.
Der Aufwand für ein einzelnes Problem kann spontan zehn mal höher ausfallen als geplant. Anders herum habe ich auch einige Features viel schneller erledigt als ich es geplant hatte.

Um diesem nicht planbaren Faktor entgegen zu wirken entschied ich mich dazu, die Code-arbeiten vorzuziehen.
Dieses Vorgehen hat nun dazu geführt dass ich sehr guten Code habe, aber leider weniger Spielinhalt als ich es mir wünschte.

Beim Lösen eines Problems war einer meiner besten Ansprechpartner das Internet.
Leider gibt dieses nicht nur eine Antwort auf eine Frage, sondern ich bekomme 100 Antworten von welchen 80 ähnliche Probleme haben, aber nur 20 das gleiche.
Von diesen 20 bekomme ich dann 3-6 verschiedene Lösungsvorschläge und muss ausprobieren, welcher am besten für mein Projekt geeignet ist.




\section{Erweiterungsmöglichkeiten}
Hätte ich mehr Zeit, wären mehr Aufgaben und mehr Interaktionen meine erste Wahl.

Weitere Möglichkeiten sähe ich im hinzufügen von mehr Waffen und kosmetischen Skins.

\subsection{Weitere Karten}
Am Anfang der Maturaarbeit hatte ich eigentlich vor, dass es in dem Spiel drei Karten geben würde.
Aus Zeitgründen wurde nur eine Vollendet.
Diese zwei Zusatzkarten wären eine grosse Möglichkeit, mehr Inhalt beizufügen.
Da der Schwerpunkt meiner Arbeit auf dem Code liegt und nicht auf Design entschied ich mich dazu, die zwei extra Karten weg zu lassen.

\subsection{Blocken}
Schläge des NPCs blocken zu könnte mit ein wenig mehr Zeit implementiert werden.
Die Animation dazu hätte ich schon.
\subsection{Weitere Verbreitung}

Mir ging die Idee durch den Kopf das Spiel online zu stellen auf Indiegame-seiten.
Da es aber im Prinzip nichts nie da gewesenes oder Bahnbrechendes ist, habe ich mich dagegen entschieden.
Für den Fall dass ich damit Geld verdienen wollen würde, könnte man mit Google ads auch Werbung schalten.
Ich halte aber von Werbung nichts, ausserdem ist dies eine Maturarbeit und kein Free-To-Play Pay-to-win 
%todo fussnote
(ein Mechanismus durch den man sich mit Mikrotransaktionen einen unfairen Vorteil im Spiel erkaufen kann) 
Spiel.

\subsubsection{Betriebssysteme}
Mit Unity kann man so gut wie jedes Betriebssystem ansteuern.
Dies führt dazu, dass ich das Spiel theoretisch auch für alle Geräte zur Verfügung stellen könnte. 
Dennoch habe ich mich dazu entschieden, es nur auf Windows auszulegen.
Warum?

Mobiltelefone haben zu kleine Bildschirme und keine Tastatur, also fallen diese Geräte weg. Dann bleiben noch die verschiedenen Betriebssysteme für Laptops oder Desktop PCs.
Von den dreien habe ich mich für Windows entschieden, weil ich selber das meistens benutze und meine \glqq Zielgruppe\grqq, also Gamer, fast ausschliesslich Windows verwenden, da schlicht die meisten Spiele nur für dieses OS zugänglich sind. Also entschloss ich mich, mich dem Teufelskreis anzuschliessen und ebenfalls nur Windows zu verwenden.

Ein macOS Testlauf ist geglückt: Es ist möglich, das Projekt als OSX version zu builden und unter dem System laufen zu lassen. Ein erster Durchlauf hat an sich einwandfrei funktioniert. Für ausgiebige Tests fehlte mir aber die Zeit.
