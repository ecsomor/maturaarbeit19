\chapter{Reflexion }
\section{Produktreflexion}
Nach nun mehr als einem halben Jahr Arbeit soll ich das Ergebnis bewerten.

Das Spiel das ich in dieser Zeit erschaffen habe beinhaltet fast alle Elemente die ich mir vorgenommen hatte.
In meinen Augen ist das Spiel jedoch noch lange nicht "vollendet", da ich immer noch mehr dazu tun könnte.
Doch dafür bräuchte ich noch mehr Zeit, und Zeit ist was wir nicht haben.
Im Grossen und Ganzen bin ich sehr jedoch sehr zufrieden. Ich konnte ein Projekt machen, das mir Spass bereitete und bei dem ich mit Leidenschaft dabei war, selbst wenn mich einige Probleme manchmal zur Weissglut treiben konnten. 

Wobei für vielen Schüler/innen wahrscheinlich gleich erging wie mir, ist dass der Konstante Druck einem den letzten Nerv zieht.
Ich habe keine Anhaltspunkte, WIE GUT meine Arbeit nun wirklich ist.
Ich weiss nicht, wie diejenigen die diese Arbeit bewerten werden drauf sind, ob sie überhaupt Programmieren können, und ob wenn sie es können, sie von meinen Fähigkeiten überhaupt beeindruckt sein werden.
Zusammengefasst ist es die Ungewissheit, die mir zu schaffen macht.

Ich habe während der Arbeit auch einige neue Dinge gelernt.
Zum Beispiel kannte ich LaTeX vorher noch nicht.
Dieses Tool gefällt mir, da die darin vorhandenen Grundfunktionen aus reinem Inhalt (alles in rein Text) gut aussehende Dokumente erstellt.

Der Umgang mit den mir schon bekannten Werkzeugen lieferte zusätzliche Erfahrung, wodurch ich diese nun noch besser beherrsche.

\section{Ziele}

gesetzte Ziele

\begin{itemize}
\item Grundgerüst (Einfaches Terrain, Lightsource, charactermodel+controller) //Implementiert
\item Waffensystem	//currently working on
\item Player HUD
\item Inventarsystem
\item NPCs
\item Skilltree
\item Savegames
\item Story
\item Quests

\end{itemize}

erreichte Ziele

\section{Arbeitsweise}


\subsection{Untertitel}

\section{Arbeitsmethodik}

\subsection{Download und Kontakt}

\subsection{Danksagung}
