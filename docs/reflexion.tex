\chapter{Schluss}

\section{Reflexion }

Nach nun mehr als einem halben Jahr Arbeit soll ich das Ergebnis bewerten.

\smallskip
Das Spiel, das ich in dieser Zeit erschaffen habe beinhaltet fast alle Elemente, die ich mir vorgenommen hatte.
In meinen Augen ist das Spiel jedoch noch lange nicht \glqq vollendet\grqq, da ich immer noch mehr dazu tun könnte.
Doch dafür bräuchte ich noch mehr Zeit, und Zeit ist, was wir nicht haben.
Im Grossen und Ganzen bin ich sehr jedoch sehr zufrieden. Ich konnte ein Projekt machen, das mir Spass bereitete und bei dem ich mit Leidenschaft dabei war, selbst wenn mich einige Probleme manchmal zur Weissglut treiben konnten. 

\smallskip
Unity ist ein Profi-Werkzeug, mit welchem teure kommerzielle Projekte durchgeführt werden, das heisst es ist sehr vielfältig und lässt den Benutzer somit so gut wie alles erstellen.
Als \glqq Amateur\grqq ist es entsprechend schwieriger, Dinge wie z.B. ein Feuer zu erstellen, da in diesem Beispiel rund 30 Parameter für die Umsetzung der Partikel zur Verfügung stehen.
In solchen Momenten war ich öfters als erwartet gezwungen, Anleitungen zu konsultieren, um mich nicht in einem Trial-and-Error Verfahren zu verlieren.

\smallskip
Wobei es vielen Schüler/innen wahrscheinlich gleich erging wie mir, ist dass der konstante Druck einem den letzten Nerv zieht. Ich habe keine Anhaltspunkte, WIE GUT meine Arbeit nun wirklich ist.
% Ich weiss nicht, wie diejenigen die diese Arbeit bewerten werden drauf % sind, ob sie überhaupt Programmieren können, und ob wenn sie es % können, sie von meinen Fähigkeiten überhaupt beeindruckt sein werden.
Die Tatsache, dass die nötige Dauer um ein Problem im Code zu lösen nicht schon vorher bekannt  und für mich nur sehr ungenau abschätzbar ist, erschwerte mir das Zeitmanagement mehr, als ich es erwartet hatte.
Zusammengefasst ist es die Ungewissheit, die mir zu schaffen machte.

\smallskip
Ich habe während der Arbeit auch einige neue Dinge gelernt.
Zum Beispiel kannte ich LaTeX vorher noch nicht.
Dieses Tool gefällt mir, da die darin vorhandenen Grundfunktionen aus reinem Inhalt (alles in aus Text) gutaussehende Dokumente erstellen.
Der Umgang mit den mir schon bekannten Werkzeugen lieferte zusätzliche Erfahrung, wodurch ich diese nun noch besser beherrsche.

\section{Download und Kontakt}

Das GitHub Repository befindet sich in\\
\url{https://github.com/ecsomor/maturaarbeit19}\\
Für Rückfragen: \href{mailto:elias.csomor@stud.ksimlee.ch}{elias.csomor@stud.ksimlee.ch}

\section{Danksagung}
Grosser Dank geht an Herrn Graf, meinen Betreuer, der mich in dem Labyrinth der Möglichkeiten und Schwerpunkte begleitet hat.
Ich bedanke mich auch bei meinen ehemaligen Betreuern der ersten 2 Maturaarbeits-Versuche, auch wenn es nicht geklappt hat.
Dank auch an die \glqq Nerds\grqq aus meiner Klasse und aus dem Mint-Labor, die mir als Spieletester geholfen oder mir die 3D Modellierung in Blender näher gebracht haben.
Schliesslich will ich meinem Vater danken, der mich bei dieser Maturaarbeit sehr unterstützt hat.