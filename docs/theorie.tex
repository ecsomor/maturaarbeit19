\chapter{Grundlagen}

\section{Werkzeuge}

\subsection{Unity 3D Programmierumgebung}
Beschreibung der Oberfläche Einsatz

\subsubsection{Scene View}
Dieses Fenster ermöglicht das Interaktive bearbeiten von Szenen und Welten.
Man kann sich in der Spielwelt überall hin und durch alle Objekte hindurch bewegen. Um Objekte zu editieren kann man sie mit X,Y und Z achsen positionieren,
skalieren und drehen.

\subsubsection{Game View}
Dieses Fenster zeigt eine Vorschau des Spiels.
Sobald Play gedrückt wird, kann man darin das Spiel spielen.
Um veränderungen auszutesten, kann man während des Spiels pausieren, veränderungen vornehmen und das Spiel mit diesen Veränderungen fortsetzen. Sobald der Spielvorgang durch ein zweites betätigen des Play buttons gestoppt wird, werden die Veränderungen wieder rückgängig gemacht.

\subsubsection{Hierarchy}
Dieses Fenster Zeigt alle in der aktuell geöffneten Szene existierenden Objekte und deren Hierarchiestruktur.

\subsubsection{Inspector}
Dieses Fenster zeigt alle öffentlichen Parameter und Komponenten des aktuell ausgewählten Objekts (GameObject) an.

\subsubsection{Project Browser}
Dieses Fenster dient der visualisierten Navigation durch die Projektdateien \textbf{(Assets)}

\subsubsection{Console}
In diesem Fenster werden alle Outputs, Warnungen und Fehlermeldungen angezeigt

\subsection{Blender}

\subsection{LaTeX}

\subsection{MonoDevelop}

\subsubsection{Oberfläche}

Scripts werden in \textbf{MonoDevelop} geschrieben und haben in dieser Umgebung auch einen Debugger


\subsection{Git}

Versionsverwaltung

\subsection{astah UML}

UML Diagramme

\section{Programmiersprache und Framework}

\subsection{C\#}

Objektorientierte Programmiersprache

\subsection{Unity Framework}

\subsubsection{Wichtigste Basisklassen}

\paragraph{GameObject}

\paragraph{MonoBehaviour}

Basis Methoden

- Start

- Update

\subsection{Physik}

\subsubsection{Schwerkraft}

\subsubsection{Kollisionen}
\label{subsubsec:collider}

\section{Glossar}

Im Folgenden möchte ich die wichtigesten Fachbegriffe kurz erläutern. \cite{unity3dglossary}

\subsubsection{Game-Engine}
Ein spezielles Programm, welches die grundlegende Funktionalität für den Ablauf und die Steuerung des Spieles zur Verfügung stellt.



