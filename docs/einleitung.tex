\chapter{Einleitung}

Was nicht viele wissen ist, dass dies tatsächlich mein dritter Anlauf einer Maturaarbeit ist.
Beim ersten Mal wollte ich eine Geschichte schreiben.
Dass das nicht geklappt hat war nicht weiter tragisch, denn ich hatte noch nicht so viel Energie darin investiert.
Beim zweiten Mal war es anders.
Ich hatte schon gut 5 Monate lang recherchiert und programmiert, um eine KS im Lee-App zu erschaffen.
Neben anderen Features wie z.B. individuelle Meldungen bei ausgefallenen Lektionen, sollte auch der persönliche Stundenplan dabei sein. 

\smallskip
Nach einer Mathematikstunde nahm mich mein damaliger Betreuer Rolf Kleiner zur Seite und überbrachte die schlechte Nachricht: Die Extranet-Server, von denen ich die Infos jeweils abholte, sollten in zwei Wochen abgeschaltet werden. Um die neuen Server benutzen zu können, müsste man sich mit der Zentrale in Zürich in Verbindung setzen. Das taten wir auch, aber nach den ersten E-Mails wurde klar, dass das ein viel zu grosser Aufwand sein würde. An dieser Stelle musste ich die Arbeit verwerfen, da es zu rechtlichen Problemen mit den Betreibern der Intranet-Version kam (dies wäre die zweite mögliche Quelle für meine Daten gewesen).

\smallskip
Bei der Wahl des neuen Themas (und somit auch indirekt des Titels) war es mir wichtig, dass der Name Programm ist.
\textbf{SimpleRPG} steht dafür, dass es ein \textbf{einfaches} Spiel ist, in welchem die \textbf{Schlüsselelemente} eines \textbf{Rollenspiels} (\textbf{RPG} = Role Playing Game) vorhanden sind.

\smallskip
Warum die Betonung auf \textbf{einfach}? Aktuell ist es der Fall, dass nahezu bei jeder Videospielentwicklung auf einen Programmierer rund zehn Grafiker kommen.
Ich als Einzelperson wollte aber mehr programmieren als Graphiken gestalten, da für mich der Schwerpunkt meiner Arbeit beim Code schreiben liegen sollte.
Ich hätte die Option gehabt, \textbf{Unity Assets}, eine Sammlung von vorgefertigten Elementen, Figuren, Umgebungselementen etc. herunter zu laden.
Diese wären aber nicht selbst gemacht und somit nicht vereinbar damit, dass ich das Spiel selber von Grund auf selbst erstellen wollte.
\paragraph{Elemente}
Ich programmiere ein Spiel, welches das Genre des RPGs auf seine \textbf{Grundbausteine} runterbricht: \textbf{Quests, Loot, Level}. Geschmückt wird das Ganze mit einer Geschichte von Invasion, Familie und Verrat.



