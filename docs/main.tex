\RequirePackage[l2tabu, orthodox]{nag}
\documentclass[a4paper,twoside,openright]{scrreprt}
\usepackage[utf8]{inputenc}
\usepackage[ngerman,english]{babel}

% Latin Modern

\usepackage{lmodern}
\usepackage[T1]{fontenc}

% Layout

\usepackage[left=2cm,right=2cm,top=2cm,bottom=2cm]{geometry}

% Mathe Packages

\usepackage{amsmath}
\usepackage{amsfonts}
\usepackage{amssymb}

% komfortable Referenzen mit \fref
\usepackage[german]{fancyref}

% Kopf- und Fusszeilen
\usepackage[automark,headsepline]{scrpage2}
\ihead[]{\headmark}
\chead[]{}
\ohead[]{\pagemark}
\ifoot[]{}
\cfoot[\pagemark]{}
\ofoot[]{}
\pagestyle{scrheadings}

\usepackage{makeidx}

% Einfügen von Bildern
\usepackage{graphicx}

% schöne Tabellen
\usepackage{booktabs}

% Hyperlinks in PDF
\usepackage{hyperref}

\titlehead{Kantonsschule im Lee \hfill Elias Csomor\\
Fachschaft Informatik \hfill Heldweg 12\\
Rychenbergstrasse 140 \hfill 8475 Ossingen\\
8400 Winterthur
}
\subject{Maturaarbeit}
\title{JustAnotherRPG}
\subtitle{Erschaffen eines Spiels von 0 aus}
\author{\texorpdfstring{Elias Csomor\\[1cm]\\[1cm] {\small Betreuer: Thomas Graf}}{Hans Muster}}
\date{\small Stand 2.9.2018}

% PDF Docinfo

\makeatletter
\hypersetup{
	pdftitle={\@title},%
	pdfsubject={\@subject},%
	pdfauthor={\@author},%
	pdfkeywords={},%
	colorlinks,%
	citecolor=black,%
	filecolor=black,%
	linkcolor=black,%
	urlcolor=black}%
\makeatother

% Real Content

\begin{document}
\selectlanguage{ngerman}
\maketitle % <- Titel setzen
\cleardoublepage
\pagenumbering{roman} % <- römische Seitennummerierung
\tableofcontents % <- Inhaltsverzeichnis
\cleardoublepage % <- neue Seite
\pagenumbering{arabic} % <- arabische Seitennummerierung

\section{Arbeitstaktik}


\subsection{About}

Ich Programmiere ein Spiel, das das Genre des RPGs auf seine Grundbausteine runterbricht: Quests, Loot, Level. Geschmückt wird das ganze mit einer Geschichte von Invasion, Familie und Verrat.

\subsection{Tools}
Ich benutze die Unity 3D engine für den Hauptteil der Arbeit. Scipts werden in MonoDevelop geschrieben.

\subsection{Vorgehensweise} Ich trage jeden Tag ein Heft in meiner Schultasche mit mir, in welchem ich Auftauchende Ideen sofort aufschreibe/aufzeichne.
Wenn ich dann wieder an meinem Computer sitze, werden diese Ideen nochmals durchgeschaut und entweder verworfen oder Implementiert.
Ich sorge dafür dass ich jede woche mindestens einmal daran Arbeite, und höre wenn ich mal begonnen habe auch nicht so schnell wieder auf.
 \subsection{Dokumentation}
Meine quellen (meistens Youtube oder Udemy tutorials) so wie der damit verbundene Fortschritt dokumentiere ich in einem TXT File.
In naher zukunft migriere ich jedoch auf Github.



\appendix % <- Anhang
\listoffigures % <- Abbildungsverzeichnis
\listoftables  % <- Tabellenverzeichnis
% \input{chapters/bibliography} % <- Literaturverzeichnis

\end{document}